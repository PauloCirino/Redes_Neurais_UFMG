\documentclass[]{article}

\usepackage[utf8]{inputenc}
\usepackage{url}
\usepackage{breakurl} 
\usepackage[breaklinks]{hyperref}
\usepackage{lipsum}
\usepackage{multicol}
\usepackage{graphicx}
\usepackage{float}
\usepackage{psfrag}
\usepackage{amssymb}
\usepackage{graphics}
\usepackage{amsmath}
\usepackage{url}
\usepackage{algorithm}
\usepackage{algpseudocode}
\usepackage{tabularx}
\usepackage{filecontents}
\usepackage{url}
\usepackage{multirow}
\usepackage{mathtools}
\usepackage{scrextend}
\usepackage[flushleft]{threeparttable}
\usepackage{subfigure}
\usepackage{textcomp}
\usepackage{tablefootnote}

\def\UrlBreaks{\do\/\do-}

\algblockdefx[Foreach]{Foreach}{EndForeach}[1]{\textbf{foreach} #1 \textbf{do}}{\textbf{end foreach}}

\providecommand{\algorithminput}[1]{
	~\\
	\textbf{Input:}\\
	\begin{tabularx}{\textwidth}{rl}#1\end{tabularx}
}
\providecommand{\algorithmoutput}[1]{
	~\\
	\textbf{Output:}\\
	\begin{tabularx}{\textwidth}{rl}#1\end{tabularx}
}

\newcommand{\Break}{\textbf{break}}


%opening
\title{\textit{Undersampling} Representativo de Classe Dominante por Fator de Qualidade Baseado em Multiplicadores de Lagrange  }
\author{Paulo Cirino}

\begin{document}

\maketitle

\begin{abstract}
	
\end{abstract}

\section{Introdução}

Esse trabalho é fundamento em um algorítimo de \textit{fuzzy clustering}, à ser publicado, que foi criado para acelerar o método \textit{Fuzzy C Means}. Ele atinge seu objetivo por meio da remoção de pontos da bordas com o auxilio de um fator de qualidade, que diz respeito a pertinência de cada amostra à todos os \textit{clusters}.

Os algorítimo de  \textit{fuzzy clustering } permitem que uma amostra de um \textit{data set} pertença, ao mesmo tempo, à múltiplos agrupamentos. O nível que uma amostra pertence a cada \textit{cluster} é tradicionalmente chamado de \textbf{pertinência $\mu_{i}(x_j)$}, que é a pertinência da amostra $x_j$ para o \textit{cluster} $i$.

A função de custo $J$, associada à problemas de \textit{fuzzy clustering}, pode ser definida em 
\ref{eqJ1}.
\begin{equation}
\label{eqJ1}
\begin{aligned}
& {\text{min}}
& & J \\
& \text{sujeito a}
& & \sum_{k=1}^c u_{ik}=1,   \quad k=1,2,...,N \\
\end{aligned}
\end{equation}

Onde $J$ é definido em \ref{eqJ2}.
\begin{equation}
\label{eqJ2}
J = \sum_{i=1}^c \sum_{k=1}^N u_{ik}^2 d_{ik}^2
\end{equation} 


Nessa situação $\mu_{ik}$, é a pertinência da amostra $k$ em relação ao centro $i$. Adotando a solução de Multiplicadores de Lagrange, a nova função de custo assume a forma descrita em \ref{eqJ22}, com derivadas parciais \ref{eqdjdl} e \ref{eqdjdu}. 

\begin{equation}
\label{eqJ22}
J = \sum_{i=1}^c \sum_{k=1}^N \left [  u_{ik}^2 d_{ik}^2 - \lambda (\sum_{m=1}^c u_{mk}-1)\right]
\end{equation}

\begin{equation}
\label{eqdjdl}
{{\partial J} \over {\partial \lambda}} = \sum_{m=1}^c u_{mk}-1 : {{\partial J} \over {\partial \lambda}} = 0 \implies \sum_{i=1}^c u_{ik}=1
\end{equation}

\begin{equation}
\label{eqdjdu}
{{\partial J} \over {\partial u_{st}}} = 2u_{st} d_{st}^2 - \lambda : {{\partial J} \over {\partial u_{st}}} = 0 \implies u_{st} = {\lambda \over {2d_{st}^2}}
\end{equation}

Assim, a equação \ref{eqLamdaT}, representa cada um dos multiplicadores de Lagrange do \textit{data set}.

\begin{equation}
\label{eqLamdaT}
\lambda_{k} = {2 \over {\sum_{j=1}^c {1 \over d_{jk}^2}}}, \quad k=1,2...,N
\end{equation}

Assim é possível definir uma medida de qualidade para cada amostra, descrita na equação, \ref{eqqk}. A medida $q_k$ de qualidade, é obtida para cada amostra $\mathbf{x}_k$ de $\mathbf{X}=\{x_i \in \mathbb{R^d} | i=1...N\}$, e representa uma medida de incerteza da pertinência $\mu_{ik}$.

\begin{equation}
\label{eqqk}
q_k = 1 - c^c \prod_{i=1}^c {1 \over \mu_{ik}}
\end{equation}

Substituindo a equação \ref{eqdjdu} em \ref{eqqk}, podemos representar $q_k$ em \ref{eqqk_lambda_a0}

\begin{equation}
\label{eqqk_lambda_a0}
q_k = 1 - {2 \over {\lambda_k}} c^c \prod_{i=1}^c d_{ik}^2
\end{equation}

É importante notar que amostras fortemente ligadas a um determinado centro, terão $q_k$ muito próximo à $1$, aquelas que estão distantes terão valores muito próximos a 0.


\section{Método}

\section{Resultados}

\section{Conclusão}

\end{document}
